
\usepackage{ctex}
\usepackage{geometry} %页面设置
\usepackage{listings} %插入代码
\usepackage{xcolor}

\usepackage{setspace} %行间距包
\usepackage{paralist} 

\usepackage{float}


%页面设置
\geometry{left=2.5cm,right=2.5cm,top=2.5cm,bottom=2.5cm}

%字体设置
\setmainfont{Times New Roman} %设置英文默认字体
\setCJKmainfont[BoldFont=SimHei]{SimSun} %设置中文默认字体为宋体

%加载其他英文字体
\newfontfamily\courier{Courier New} %使用Courier New字体

%加载其他中文字体
\setCJKfamilyfont{hwxk}{FangSong}%使用仿宋字体
\newcommand{\stxk}{\CJKfamily{hwxk}}

%段落格式
\parindent 2em   %段首缩进

\let\itemize\compactitem 
\let\enditemize\endcompactitem 
\let\enumerate\compactenum 
\let\endenumerate\endcompactenum 
\let\description\compactdesc 
\let\enddescription\endcompactdesc

%代码块设置
\lstset{
  backgroundcolor=\color{white},   % choose the background color; you must add \usepackage{color} or \usepackage{xcolor}
  basicstyle=\footnotesize\courier,% the size of the fonts that are used for the code
  breakatwhitespace=false,         % sets if automatic breaks should only happen at whitespace
  breaklines=true,                 % sets automatic line breaking
  captionpos=b,                    % sets the caption-position to bottom
  commentstyle=\color{green},      % comment style
  deletekeywords={...},            % if you want to delete keywords from the given language
  escapeinside={\%*}{*)},          % if you want to add LaTeX within your code
  extendedchars=true,              % lets you use non-ASCII characters; for 8-bits encodings only, does not work with UTF-8
  frame=shadowbox, 				% adds a frame around the code
  rulesepcolor=\color{red!20!green!20!blue!20},                   
  keepspaces=true,                 % keeps spaces in text, useful for keeping indentation of code (possibly needs   columns=flexible)
  keywordstyle=\color{blue},       % keyword style
  otherkeywords={*,...},           % if you want to add more keywords to the set
  numbers=left,                    % where to put the line-numbers; possible values are (none, left, right)
  numbersep=10pt,                   % how far the line-numbers are from the code
  numberstyle=\tiny\color{gray},   % the style that is used for the line-numbers
  rulecolor=\color{black},         % if not set, the frame-color may be changed on line-breaks within not-black text (e.g. comments (green here))
  showspaces=false,                % show spaces everywhere adding particular underscores; it overrides 'showstringspaces'
  showstringspaces=false,          % underline spaces within strings only
  showtabs=false,                  % show tabs within strings adding particular underscores
  stepnumber=1,                    % the step between two line-numbers. If it's 1, each line will be numbered
  stringstyle=\color{mauve},       % string literal style
  tabsize=2,	                      % sets default tabsize to 2 spaces
  title=\lstname,                  % show the filename of files included with \lstinputlisting; also try caption instead of title
  columns=flexible,
  xleftmargin=4em,xrightmargin=2em
}
